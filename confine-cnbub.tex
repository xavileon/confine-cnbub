\documentclass[conference]{IEEEtran}
\usepackage[utf8]{inputenc}
\usepackage{cite}

\ifCLASSINFOpdf
  \usepackage[pdftex]{graphicx}
  \graphicspath{{.}{images/}{pdf/}{jpeg/}}
  \DeclareGraphicsExtensions{.pdf,.jpeg,.png}
\else
  \usepackage[dvips]{graphicx}
  \graphicspath{{.}{images/}{../eps/}}
  \DeclareGraphicsExtensions{.eps}
\fi

\usepackage[caption=false,font=footnotesize]{subfig}

\usepackage{url}

% correct bad hyphenation here
\hyphenation{op-tical net-works semi-conduc-tor}


\begin{document}

\title{CONFINE: the testbed for community networks}


% author names and affiliations
% use a multiple column layout for up to three different
% affiliations
\author{\IEEEauthorblockN{Xavier León, Felix freitag and Leandro Navarro}
\IEEEauthorblockA{Departament d'Arquitectura de Computadors\\
Universitat Politècnica de Catalunya\\
Barcelona, Spain\\
\texttt{\{xleon, felix, leandro\}@ac.upc.edu}}
\and
\IEEEauthorblockN{Axel Neumann and Iván Vilata}
\IEEEauthorblockA{Pangea.org\\
NGO Internet Solidario\\
Barcelona, Spain\\
\texttt{neumann@cgws.de, ivan@pangea.org}}}

\maketitle


\begin{abstract}
The abstract goes here. bla bla bla?
\end{abstract}

\section{Introduction}

- on testbed for researchers in Community networks.
(maybe there are parts in DoW from Confine that can be copied)

1.1 why is research in CN important (Confine-Brochure in www.confine-project.eu)
  - to support growth of CNs (scalability, sustainability)
  - future internet and digital agenda

1.2 what research in CN (CN shortomings)
  - howto impove privacy
  - networking robustness (eg routing optimization, cross-layer optimizations)
  - efficient content distribution
  - network policies evaluation/optimization
  - heterogenous physical networks (wireless, broadband fiber)
  - node robustness (eg automatic reconfiguration, deployment, failure handling...)

1.3 why difficult to do reasearch? (Researcher shortcomings)
   - there are no real-life CN testbeds (only small lab deployments)
   - no existing implementations available
   - community driven effort (not a controlled lab)

\section{Related Work}

I cite here planetlab \cite{planetlab}

= Other testbeds, what they offer (PlanetLab (Xavi knows well), and wireless
testbeds that are known from papers (Berlin testbeds, Nitos, ...?)

\section{Challenges and Requirements}


2. what challenges it poses to develop a testbed on top of it. ( 3 pages )
  - keep stability and security
  - keep CN privacy
  - researcher affiliation != testbed management ?!= node ownership
  - usablitiy (simple API despite system complexity)
  - not CPU powerful devices (cheap)
  - remote locations
  - connection instability, variable connectivity and available bandwidth
  - cross-CN communication (testbed spanning several CNs)
  - lack of IPv6 support in existing CNs (tinc workaround)

\section{CONFINE architecture}

\subsection{Long-term vision}

3.1 architecture (envisioned arch:node)
-integration in community networks
\subsection{Current state}
3.2 current state
- implementation (target CONFINED)
- note on IPv6 overlay
\subsection{Next steps}
3.3 next steps
- 2n year implementation (between CONFINED and envisioned)

\subsection{Discussion}

Compare the Confine node architecture's features/potential ... with that of
related work.
(there was a mail from Axel some weeks ago: He mentioned the unique features of
Confine: parallel experiments (other wireless networks are not virtualized) +
long term studies (since other testbeds have dedicated nodes, users will not be
allowed to block them for too long I guess ...)

%\begin{figure}[!t]
%\centering
%\includegraphics[width=2.5in]{myfigure}
% where an .eps filename suffix will be assumed under latex, 
% and a .pdf suffix will be assumed for pdflatex; or what has been declared
% via \DeclareGraphicsExtensions.
%\caption{Simulation Results}
%\label{fig_sim}
%\end{figure}


%\begin{figure*}[!t]
%\centerline{\subfloat[Case I]\includegraphics[width=2.5in]{subfigcase1}%
%\label{fig_first_case}}
%\hfil
%\subfloat[Case II]{\includegraphics[width=2.5in]{subfigcase2}%
%\label{fig_second_case}}}
%\caption{Simulation results}
%\label{fig_sim}
%\end{figure*}



\section{Conclusion}
The conclusion goes here.


\section*{Acknowledgment}

The authors would like to thank...





% trigger a \newpage just before the given reference
% number - used to balance the columns on the last page
% adjust value as needed - may need to be readjusted if
% the document is modified later
%\IEEEtriggeratref{8}
% The "triggered" command can be changed if desired:
%\IEEEtriggercmd{\enlargethispage{-5in}}

\bibliographystyle{IEEEtran}
\bibliography{confine-cnbub}

\end{document}


